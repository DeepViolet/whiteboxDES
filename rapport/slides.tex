\documentclass{beamer}

\mode<presentation>
{
  \usetheme{default}      % or try Darmstadt, Madrid, Warsaw, ...
  \usecolortheme{default} % or try albatross, beaver, crane, ...
  \usefonttheme{default}  % or try serif, structurebold, ...
  \setbeamertemplate{navigation symbols}{}
  \setbeamertemplate{caption}[numbered]
} 

\usepackage[french]{babel}
\usepackage[utf8x]{inputenc}

\author{David Wong
  \and Jacques Monin
  \and Hugo Bonnin}

\title{Implémentation et Analyse d'une White-box du DES}

\institute{Université de Bordeaux}

\date{2014}

\begin{document}

\begin{frame}
  \titlepage
\end{frame}


\begin{frame}{Plan}
  \tableofcontents
\end{frame}

\section{A quoi ça sert ?}

\begin{frame}{A quoi ça sert ?}

\end{frame}

\subsection{Man In The Middle}

\begin{frame}

\end{frame}

\subsection{Man At The End}

\subsection{Exemples}

\subsubsection{Backdoors}

\subsubsection{DRMs}

\section{Whitebox}

\subsection{Définition}

\subsection{DES}
\begin{frame}{Algorithme DES}
\begin{itemize}
\item Le but est de transformer toutes ces opérations
\begin{figure}[h]
\centering
%\caption{Schéma d'une itération de DES}
\label{fig:DES-round}
\end{figure}
\end{itemize}
\end{frame}

\subsection{Github}

\section{Concepts}

\subsection{Partial Evaluation}
\begin{frame}{Partial evaluation}
\begin{itemize}
\item Regrouper le XOR entre le bloc et la clé avec l'opération de substitution.
\item On peut ensuite précalculer toutes les sorties possibles de cette opération.
\item Les tables créées sont les seules du programme à être modifiées lorsqu'une nouvelle clé est utlisée.
\end{itemize}
\end{frame}


\subsection{Tabularization}

\subsection{Input/Output Encoding}

\section{Concepts secondaires}

\subsection{Randomization}

\subsection{Mixing Bijection}

\subsection{Bypass}

\subsection{Combined Function Encoding}

\subsection{Split-Path Encoding}

\subsection{External Encoding}

\section{Conclusion}


\end{document}