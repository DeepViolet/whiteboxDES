\documentclass[a4paper,12pt]{article}

\usepackage{graphicx}
\usepackage{mathtools}

\author{David Wong
  \and Jacques Monin
  \and Hugo Bonin}

\title{Whitebox}

\begin{document}

\maketitle

\section{Introduction}

\subsection{DRM}

\subsection{Problèmes}

l'attaquant à accès à la mémoire et donc il peut facilement récupérer la clef en analysant l'exécution du programme.\\
De base il peut aussi choisir ce qu'il envoit au programme et voir comment le programme l'encrypte ou le décrypte (chosen plaintext attack) (cas d'une blackbox)\\
Whitebox: l'attaquant a encore plus de possibilité puisqu'il controle l'environement: accès à la mémoire, trace, breakpoints,...

\subsection{Solution}

but : Rendre l'extraction de la clef impossible

\section{DES}

\subsection{Pourquoi DES?}

\section{Concepts}

\subsection{Look up tables}

\section{Notre implémentation}

\subsection{Les fonctions de base}

\end{document}
