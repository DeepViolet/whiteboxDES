\documentclass[a4paper,12pt]{article}

\usepackage{graphicx}
\usepackage[utf8]{inputenc}
\usepackage[T1]{fontenc}
\usepackage[francais]{babel}
\usepackage{mathtools}

\author{David Wong
  \and Jacques Monin
  \and Hugo Bonnin}

\title{Whitebox}

\begin{document}

\maketitle
\newpage
\tableofcontents
\newpage


\section{Introduction}
	
\subsection{Utilisation}

\subsection{Problèmes}

l'attaquant à accès à la mémoire et donc il peut facilement récupérer la clef en analysant l'exécution du programme.\\
De base il peut aussi choisir ce qu'il envoie au programme et voir comment le programme l'encrypte ou le décrypte (chosen plaintext attack) (cas d'une blackbox)\\
Whitebox: l'attaquant a encore plus de possibilité puisqu'il contrôle l'environnement: accès à la mémoire, trace, breakpoints,...

\subsection{Solution}

Le but est de rendre l'extraction de clef impossible. Pour cela, nous allons prendre un algorithme connu : DES et y appliquer de la tabularisation ainsi que de la délinéarisation qui sont des concepts que nous expliquerons ultérieurement.

\newpage
	
\section{DES}

DES est un cipher inventé en xxxx ...

\subsection{Pourquoi DES?}

\subsection{Comment marche DES}

...


\newpage

\section{Principes et Concepts}

Plusieurs concepts ont étés introduits pour les Whitebox par Chow et cie dans leur article. On peut généralement les diviser en deux parties : l'obfuscation qui cherche à rendre la compréhension du programme plus difficile, et
 l'encryption des entrées/sorties qui cherche à rendre la recherche exhaustive fastidieuse.

\subsection{Partial Evaluation}

On cherche à cacher la clef. La première étape est de pré-calculer toutes les opérations qui sont déjà connu dans notre cas. Puisque dans une whitebox la clef est déjà connu, on pré-calcule toutes les sous-clefs et on cache l'étape du XOR avec ces sous-clefs dans des Lookup tables.

Les lookup tables sont des tableaux qui listent les sorties possibles en fonction des entrées possibles.

\subsection{Tabularizing}

Cette seconde étape consiste à transformer toutes les opérations en Lookup tables.

fonctionnement :

0101 := 5 => L(5) = ?

L = { 0000 => 1, 0001 => 2, ... } comme un tableau



\subsection{Randomization and De-linearization}

aussi appelé I/O-blocked Encoding :

Une fois que toutes les opérations ont étés codés en Lookup tables, il est encore facile de retrouver les matrices qui forment toutes ces transformations et donc d'en extraire la clef. Une façon de rendre ce travail trop fastidieux est d'encoder les entrées et sorties de ces Lookup tables avec des bijections (pour pouvoir les annuler d'une Lookup table à une autre).
Par exemple, si $f_1$ et $f_2$ sont deux lookup tables se suivant ($output = f_2 circ f_1(input)$). On peut rajouter une bijection $E$ entre les deux comme ceci : $f_2 \circ E^-1 \circ E \circ f_1(input)$.

\section{Principes et concepts secondaires}


\subsection{Mixing Bijection}

La création des Lookup tables dans la tabularization se fait à partir de matrices représentant les opérations du cipher. Souvent ces matrices contiennent beaucoup plus de 0 que de 1 ce qui rend les lookup tables trop simples et ce qui crée donc un nouveau problème. Pour éviter ce genre de problème on crée deux opérations au lieu d'une. La première composé d'une bijection. La deuxième composé de son inverse multiplié par l'opération qu'on veut complexifier.\\
Soit $M_1$ la matrice d'une des opérations que l'on transformera en lookup table plus tard. On peut la multiplier par une autre matrice $G$ choisie pertinament pour augmenter le bruit dans $M_1$ (un nombre de 0 et de 1 qui parait aléatoire). De fait que l'opération devienne alors $G \cdot M_1$. Il faut bien sur annuler $G$ ensuite en créant une seconde opération $G^(-1)$.\\
L'opération devient alors : $G^(-1) \cdot (G \cdot M_1)$ où $G \cdot M_1$ est pré-évalué comme une unique opération.

\subsection{By-Pass Encoding}

En général, on veut cacher ce qu'une opération fait. L'idée du bypass est d'élargir la taille de son entré et la taille de sa sortie avec des bits inutiles.

\subsection{Combined Function Encoding}

Lorsque deux opérations sont évalués en même temps, on peut les faire évaluer une entrée composée de leurs deux entrées respectives $(P||Q)(input_P||Input_Q)$.

\subsection{Split-Path Encoding}

même idée que le bypass ?

\section{Implémentation}

Les concepts introduits précédemment sont très théoriques et il faut réfléchir un peu pour les mettre en pratique. Voici notre interprétation:

\subsection{Partial Evaluation}

On crée un programme prenant une clef en entrée et générant toutes les lookup tables dans un fichier qui sera utilisé pour compiler la whitebox plus tard.

\subsection{Tabularizing}

Problème :
On ne peut pas faire des lookup tables trop grosses (3.2) ca augmente exponentiellement.

pour 8 bits d'entrées il y a 2^8 possibilités. Donc la taille de la lookuptable est de 2^8 octets. ou 2^8 * sizeof(int) puisqu'on utilise int dans notre implémentation.
Pour 16 bits d'entrées, il y a 2^16 possibilités. Donc la taille de la lookup table est de 2^16 octets...

On voit que ca augmente exponentiellement. Il faut donc prendre des lookup tables petites et ne pas faire une énorme lookup table pour notre vecteur de 96 bits.

On crée donc des \textbf{réseaux} de lookup tables.

----

Dans ce même programme on transforme toutes les opérations en \textbf{matrices}.

Chaque matrice est découpé en \textbf{blocs de sous-matrices} et les multiplications des sous-matrices au vecteur associé sont transformées en Lookup tables. On utilise des blocs de matrice assez conséquent pour éviter les blocs nuls (il y a beaucoup de zéros dans ces matrices).

Ensuite tous les XOR sont transformés en Lookup tables aussi qu'on appellera des XOR tables.

Après cette étape toutes les opérations ont étés codés en lookup tables

Par exemple, nous allons décomposer une matrice 16*16 en sous-matrices de 8x4. (selon Link et cie, c'est plus secure).

Voici l'opération sans la décomposition

\begin{figure}[h]
\begin{verbatim}
           .----.     .-------------------.     .----.
           |    |     |                   |     |    |
           |    |     |                   |     |    |
           |    |     |                   |     |    |
           | Y0 |  =  |         M         |  x  | X0 |
           |    |     |                   |     |    |
           |    |     |                   |     |    |
           |    |     |                   |     |    |
           '----'     '-------------------'     '----'

\end{verbatim}
\caption{Avant décomposition en sous-matrice}
\label{fig:ascii-box}
\end{figure}

\clearpage

Voici l'opération après la décomposition en sous-matrice

\begin{figure}[h]
\begin{verbatim}

           .----.     .---------.---------.     .----.
           |    |     |    A    |    B    |     |    |
           | Y0 |     .---------.---------.     | X0 |
           |    |     |    C    |    D    |     |    |
           .----.  =  .---------.---------.  x  .----.
           |    |     |    E    |    F    |     |    |
           | Y1 |     .---------.---------.     | X1 |
           |    |     |    H    |    I    |     |    |
           '----'     '---------'---------'     '----'
           
\end{verbatim}
\caption{Après décomposition en sous-matrice}
\label{fig:ascii-box}
\end{figure}

\clearpage

Chaque sous-matrice va être la source d'une nouvelle lookup table.

Cette lookup table va être créée par le multiplication d'une sous-matrice avec toutes les possibilités d'input.

Elle aura donc $2^8$ = 256 entrées possibles et $2^4$ = 16 sorties possibles.

Ainsi pour une matrice (m,n), et pour une sous-matrice ($m_{1}$,$n_{1}$), il y aura     $\frac{m*n}{m_{1}*n_{1}}$ lookup tables créées qui prendront $(2^n_{1})$ valeurs et pourront engendrer $(2^m_{1})$ sorties.

\subsection{Randomization}

Pour compliquer la compréhension du programme on peut mélanger l'ordre dans lequel on envois les bits à notre programme. Puisqu'on a les matrices $M_1, M_2 et M_3$, il suffit de les multiplier par d'autres matrices de mélange avant et après.

Par exemple le premier round représenté par notre implémentation :

$M_1 . M_2 . data$

data étant un vecteur de 64bits

peut se remplacer par

$(M_1 . R_1^-1) . (R_1 . M2 ) . data$

* De-linearization :

Une fois que toutes les opérations sont transformés en lookup tables. On rajoute des encodages à ces lookup tables.

Par exemple $L_1$ et $L_2$ deux lookup tables qui se suivent.

$L_1 \circ L_2$

on crée une clef pour $L_1$ et on XOR cette clef à ses sorties. Ce qui fait que $L_1$ ressemble sous le manteau à $output = (L_1(input) \oplus k_1)$

et on décrypte cette encryption dans $L_2$ en faisant un XOR dans ses inputs ($L_2$ ressemble alors à output = $L_2(input \oplus k_1).$

Si $L_1$ sort un octet, et $L_2$ prend un octet en entrée. Alors on a un encodage sur les 256 possibilités de la Lookup table. Ce qui est assez facile à retrouver avec une recherche exhaustive. C'est pourquoi on crée un nombre de Lookup tables différentes pour les XORs assez important pour rendre cette recherche exhaustive inefficace. servir

\newpage

\section{Notre implémentation}

On utilise DES
petite intro

\subsection{Traduction des concepts}

\subsubsection{Bypass}

puisqu'on a, au maximum un état de 96bits dans la mémoire, on va garder cet état de 96 bits tout le long et utiliser les bits qui ne servent à rien comme bypass.

\subsection{Différentes étapes}

\subsubsection{Étape préliminaire}

Avant que les 16 rounds ne s'exécutent, il est nécessaire que les bits d'input subissent une permutation initiale suivie d'une expansion de l'input de 64 bits en 96 bits qui permettra le bypass.
Ces deux opération seront réalisées par une matrice M1 qui ne sera utilisée qu'une fois juste avant les 16 rounds et qui réalisera ceci :

\begin{figure}[h]
\begin{verbatim}

            96b           96x64b           64b
           .----.     .-------------. 
           |    |     |             |     .----.
           |    |     |             |     |    |
           |    |     |             |     |    |
           |    |     |             |     |    |
           | Y0 |  =  |     M1      |  x  | X0 |
           |    |     |             |     |    |
           |    |     |             |     |    |
           |    |     |             |     |    |
           |    |     |             |     '----'
           '----'     '-------------'

\end{verbatim}
\caption{M1 avant décomposition en sous-matrice}
\label{fig:ascii-box}
\end{figure}

\clearpage

Nous décomposons M1 en sous-matrice de taille 4x8. Nous multiplions chacune de ces $\frac{96*64}{4*8}$ = 288 sous-matrices par les $2^8$ = 256 différentes possibilités ayant $2^4$ = 16 résultats possibles.


\subsubsection{Étape 1 à Étape 2}

Nous devons convertir ce fonctionnement :

\begin{figure}[h]
\begin{verbatim}
			
             32b               48b              16b
           ************** ********************* ********
state 1:   *     L(r)   * *       X(r)        * * r(r) *
           ************** ********************* ********
                 |                |      |         |
                 |                v      |         |
                 | *********    .....    |         v
                 | * sK(r) *--> . + .    |    .-------.
                 | *********    .....    '-->(  Merge  )
                 |                |           '-------'
                 |                v               |
                 |         .-------------.        |
                 |          \     S     /         |
                 |           '---------'          |
                 |                |               |
            32b  v                v 32b     32b   v
           ************** *************** ***************
state 2:   *    L(r)    * *    Y(r+1)   * *     R(r)    *
           ************** *************** ***************

\end{verbatim}
\caption{Avant Tabularisation}
\label{fig:ascii-box}
\end{figure}		

\clearpage

En celui-ci :

\begin{figure}[h]
\begin{verbatim}

            *********************************************
state 1:    *          state 1 (12 x 8 = 96 bits)       *
            *********************************************
               |      |      |                       |
               v      v      v                       v
            .-----..-----..-----.                 .-----.
            | T0  || T1  || T2  |       ...       | T11 |
            '-----''-----''-----'                 '-----'
               |      |      |                       |
               v      v      v                       v
            *********************************************
state 2:    *              state 2 (96 bits)            *
            *********************************************			
			
\end{verbatim}
\caption{Après Tabularisation}
\label{fig:ascii-box}
\end{figure}		
		
\clearpage		
		
Pour ce faire, nous allons calculer 12 Lookup Tables qui prendront 8 bits chacun ce qui recouvrira les 96 bits d'input.

Il y a :

8 lookup tables non linéaires qui permettent le xor avec la clef et la substitution en les précalculant.


4 lookup tables linaires qui nous serviront à bypasser les bits qui ne subissent pas de calculs.		
		
\subsubsection{Étape 2 à Étape 3}

\begin{figure}[h]
\begin{verbatim}

           ************** *************** ***************
state 2:   *    L(r)    * *    Y(r+1)   * *     R(r)    *
           ************** *************** ***************
                 |                  |           |
                 v                  |           |
               .....    .--------.  |           |
               . + .<---|    P   |<-'           |
               .....    '--------'              |
                |                               |
            32b '----------------------------------.
                                    |           |  |
                .-------------------|-----------'  |
                |               32b v              v 32b
                |               .-------.       .------.
                |              /  E-box  \     ( Select )
                |  32b        '-----------'     '------'
                |                   |              |
                v               48b v              v 16b
           ************** ********************* ********
state 3:   *   L(r+1)   * *       X(r+1)      * *r(r+1)*
           ************** ********************* ********

\end{verbatim}
\caption{Avant Tabularisation}
\label{fig:ascii-box}
\end{figure}	

\newpage

\subsubsection{Étape finale}

L'étape finale va consister à ignorer les bypass (ie, ignorer r(r+1)), inverser la dernière expansion réalisée par M2, échanger L et R en cas de décryptage et faire la permutation finale.

Toutes ces étapes vont être réalisée grâce à une matrice M3 qui va être décomposée en sous-matrice et traduite en Lookup table.
\section{Conclusion}
\newpage
\section{Bibliographie}


\end{document}
